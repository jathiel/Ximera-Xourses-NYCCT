\documentclass{ximera}
\title{Loops and Functions}
\begin{document}
\begin{abstract}
The goal is to learn to write some basic python programs.
\end{abstract}
\maketitle

\section{Basics}

In this section we introduce loops and function definitions.

	\begin{enumerate}
        	\item The Python syntax for a for loop is as follows:
	\end{enumerate}
\begin{verbatim}
==============================

for variable in iterable:
        code

==============================
\end{verbatim}

In the above code, \verb|for| indicates the start of a for loop, \verb|variable| is just a variable name of your choice, \verb|in| is a keyword, and \verb|iterable| is a Python object that can be iterated over, like a list. This Python code almost reads as an English statement. It is saying, for each item in \verb|iterable|, whose value we assign to \verb|variable| each time, perform the instructions given by \verb|code|.

The example below prints the values \verb|0| to \verb|9|.

\begin{sageCell}
for i in range(10):
        print(i)
\end{sageCell}

	\begin{enumerate}
		\item[(b)] Thy Python syntax for a while loop is as follows: 
	\end{enumerate}
\begin{verbatim}
==============================

while condition:
        code

==============================
\end{verbatim}

In the above code, \verb|while| indicates the start of the while loop and \verb|condition| is the statement that is checked to see if the code in the while loop should be exectued (the while loop will end whenever \verb|condition| is false). Note that for a while loop, if variables related to \verb|condition| are not updated within the loop, it is possible to create an infinite loop.

In many cases for and while loops can be used to execute the same computation. Typically a for loop is used if it is known in advance how many iterations are going to be needed to complete some computation. A while loop, on the other hand, is often used when the number of iterations needed is not obvious, but a stopping condition is clear.

The example below prints the values of \verb|0| to \verb|9|.

\begin{sageCell}
n = 0
while n < 10:
        print(n)
        n += 1
\end{sageCell}

	\begin{enumerate}
		\item[(c)] If/Else statements help direct your program if there are multiple cases and use the following syntax:
\begin{verbatim}
==============================
if condition:
        code1
elif:
        code2
else:
        code3
==============================
\end{verbatim}

The example below computes the absolute value of \verb|x|.
	\end{enumerate}

\begin{sageCell}
x = -5
	
if x==0:
        print(0)
elif x>0:
        print(x)
else:
        print(-x)
\end{sageCell}

	\begin{enumerate}
        	\item[(d)] A Python function will have the following syntax:
	\end{enumerate}

\begin{verbatim}
==============================
def functionName(inputs):
        code
        return variable
==============================
\end{verbatim}

The key word \verb|def| denotes the beginning of a function definition. Using the same guidelines for variable names, \verb|functionName| is your choice for the name of the function and will be used to `call' it as needed. A collection of comma separated values in \verb|inputs| will act as the inputs to the function.  The portion labeled \verb|code| is simply the set of instructions to be completed whenever the function is called. Finally, we have the line with the keyword \verb|return|. So far, we have been using \verb|print| to see the result of a calculation on the screen. In the case of a function, we will likely want the resulting output to be used in {\em another} calculation. The keyword \verb|return| provides us with this ability. Any \verb|variable| after \verb|return| represents the output value of the function and can be assigned to another variable or passed into another function without displaying the value on the screen. The keyword \verb|print|, on the other hand, prints the value to the screen but does not provide us with any data that can be assigned elsewhere. When using \verb|print|, the computer essentially forgets what it just computed.

Below is an example of the function that computes the square of any number \verb|n|.

\begin{sageCell}
def square(n):
        return n*n
\end{sageCell}

\section{Problems}

    \begin{enumerate}
    \item Write a function that computes $x^2+\sin(x)$.
\begin{sageCell}
# Use this cell to solve the problem.
\end{sageCell}

    \item Write a function that computes the derivative of $x^{3/2}$.
\begin{sageCell}
# Use this cell to solve the problem.
\end{sageCell}

    \item Write an if/else statement that returns a 1 if a number $n$ is positive, 0 if $n$ is zero, and -1 if $n$ is negative.
\begin{sageCell}
# Use this cell to solve the problem.
\end{sageCell}

    \item Write a for loop that computes ``$1+2+3+\cdots+100$".
\begin{sageCell}
# Use this cell to solve the problem.
\end{sageCell}

    \item Write a function that takes in a positive integer $n$ as an input and returns the sum ``$1+2+3+\cdots+n$".
\begin{sageCell}
# Use this cell to solve the problem.
\end{sageCell}

    \item Write a for loop that counts the number of e's in the string ``new york city college of technology".
\begin{sageCell}
# Use this cell to solve the problem.
\end{sageCell}

\end{enumerate}
\end{document}
