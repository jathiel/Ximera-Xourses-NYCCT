\documentclass{ximera}
\title{Boolean Variables and Logical Operators}
\begin{document}
\begin{abstract}
The goal is to understand how Boolean variables and logical operators work in Python.
\end{abstract}
\maketitle

\section{Boolean variables}

A Boolean variable is either true or false. In Python, these are written in the following way:

\begin{verbatim}
==============================
p = True
q = False
==============================
\end{verbatim}

Boolean values appear when doing comparisons. Try the following:

\begin{sageCell}
print(3 == 2)
print(3 > 2)
\end{sageCell}

Boolean values also determine the behavior of if/else statements and while loops.
\begin{sageCell}
i = 0
while i < 10:
        print(i)
        i += 1
\end{sageCell}

\begin{sageCell}
if True:
        print('Hi!')
else:
        print('Bye!')
\end{sageCell}
    
\section{Logical Operators}

\begin{enumerate}
    \item The negation operator:
\begin{verbatim}
==============================
not p
==============================
\end{verbatim}

    \item The \verb|AND| operator:
\begin{verbatim}
==============================
p and q
==============================
\end{verbatim}

    \item The \verb|OR| operator:
\begin{verbatim}
==============================
p or q
==============================
\end{verbatim}
\end{enumerate}

\section{Problems}

    \begin{enumerate}
    \item With $p$ true and $q$ false, compute the truth value of the following:
    \begin{enumerate}
        \item $p\vee q$
        \item $p\to q$ (Hint: There is no conditional operator, so use an equivalent form)
        \item $(p\to q)\wedge \neg q$
        \item $p\to (q\wedge \neg q)$
    \end{enumerate}
\begin{sageCell}
# Use this cell to solve the problem.
\end{sageCell}

    \item Use the code in the SageCell below that constructs truth tables for the following problems:
        \begin{enumerate}
            \item Construct a truth table for $\neg p \vee q$.
            \item Construct a truth table for $(p \vee q) \vee r$.
            \item Construct a truth table that shows that $\neg(p \vee q) \equiv \neg p \wedge \neg q$.
        \end{enumerate}
\begin{sageCell}
# Use this cell to solve the problem.
def table21():
        listComb = [[True,  True],
	            [True,  False],
		    [False, True],
		    [False, False]]

statement = 'not (p or (not p and q))'

firstLine = '{:^6} | {:^6}| {}'.format('p','q',statement+' ')
print(firstLine)
print('-'*len(firstLine))

for p,q in listComb:
        myString = "{:^6} | {:^6}| {:^" + str(len(statement)) + "}"
        nextLine = myString.format(p,q, not (p or (not p and q)))
        print(nextLine)
    
def table22():
        listComb = [[True,  True],
                    [True,  False],
		    [False, True],
		    [False, False]]

  statement1 = 'not (p or (not p and q))'
  statement2 = 'not p and not q'

  firstLine = '{:^6} | {:^6} | {} | {}'.format('p','q',statement1,statement2)
  print firstLine
  print '-'*len(firstLine)

  for p,q in listComb:
    myString = "{:^6} | {:^6} | {:^" + str(len(statement1)) + "} | {:^" + str(len(statement2)) + "}"
    nextLine = myString.format(p,q, not (p or (not p and q)), not p and not q)
    print nextLine
    
def table31():
  listComb = [[True,  True,  True],
              [True,  True,  False],
              [True,  False, True],
              [True,  False, False],
              [False, True,  True],
              [False,  True,  False],
              [False,  False, True],
              [False,  False, False]]

  statement = 'p and q and r'

  firstLine = '{:^6} | {:^6} | {:^6} | {}'.format('p','q', 'r',statement+' ')
  print firstLine
  print '-'*len(firstLine)

  for p,q,r in listComb:
    myString = "{:^6} | {:^6} | {:^6} | {:^" + str(len(statement)) + "}"
    nextLine = myString.format(p,q,r, p and q and r)
    print nextLine
    
def table32():
        listComb = [[True,  True,  True],
              [True,  True,  False],
              [True,  False, True],
              [True,  False, False],
              [False, True,  True],
              [False,  True,  False],
              [False,  False, True],
              [False,  False, False]]

statement1 = 'p and q and r'
statement2 = 'p and q and r'

firstLine = '{:^6} | {:^6} | {:^6} | {} | {}'.format('p','q', 'r',statement1,statement2)
print(firstLine)
print('-'*len(firstLine))

for p,q,r in listComb:
        myString = "{:^6} | {:^6} | {:^6} | {:^" + str(len(statement1)) + "} | {:^" + str(len(statement1)) + "}"
        nextLine = myString.format(p,q,r, p and q and r,p and q and r)
        print(nextLine)

# Remove hash tag for whatever line is desired

# 2 variables, 1 statement
table21()

# 2 variables, 2 statements
#table22()

# 3 variables, 1 statement
#table31()

# 3 variables, 2 statements
#table32()
\end{sageCell}
\end{enumerate}
\end{document}
