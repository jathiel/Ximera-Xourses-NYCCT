\documentclass{ximera}  
\title{If/Else Statements}  
\begin{document}  
\begin{abstract}  
We introduce if/else statements in Python.
\end{abstract}  
\maketitle

\section{If/Else}

In an earlier section we saw that in order to compute the absolute value of a real number $x$, we first needed to determine if it satisfied a certain condition, namely, is it the case that $x>0$? The answer to this question then determines the set of instructions to follow. We can reformulate our algorithm for computing $|x|$ as two statements in the following way:

For any real number $x$, if $x>0$, then $|x|=x$. Else (or otherwise), $|x|=-x$.

The `if' portion of the statement explains what to do if the given condition is satisfied, while the `else' portion explains what to do if the given condition is not satisfied. This can be easily mapped to the flowchart for computing $|x|$ below.

\begin{center}
	\includegraphics{absalgo.png}
\end{center}

In Python we can implement this short algorithm using the following syntax:

\begin{verbatim}
if (condition):
        code1
else:
        code2
\end{verbatim}

The term \verb|condition| indicates the condition we wish to check. If the condition evaluates to \verb|True|, then the instructions given by \verb|code1| are followed. If the condition evaluates to \verb|False|, then the instructions given by \verb|code2| are followed. Using the template above, we can now write the necessary Python code to compute $|x|$.

\begin{verbatim}
if x > 0:
        abs = x
else:
        abs = -x
\end{verbatim}

The above block of code creates a variable \verb|abs| that is assigned the absolute value of $x$ for any given $x$. The SageCell below shows how to use this code in practice.

\begin{sageCell}
x = -3            # change this value and evaluate the cell to compute |x| for other x values

if x > 0:
	abs = x
else:
	abs = -x

print(abs)        # this line prints the value of abs to show that the variable has been assigned the correct value
\end{sageCell}

Note that in Python, whitespace (indentation) determines which lines of code belong to a particular code block. For example, the following code differs from that in the SageCell above in that the \verb|print| statement is only evaluated if $x\leq 0$.

\begin{sageCell}
x = -3

if x > 0:
	abs = x
else:
	abs = -x
        print(abs)
\end{sageCell}






\begin{problem}
\end{problem}

\begin{problem}
\end{problem}

\end{document}
