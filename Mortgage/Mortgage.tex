\documentclass{ximera}  
\title{Mortgage Problems}  
\begin{document}
\begin{abstract}  
We introduce some mortgage rate problems as an application.
\end{abstract}  
\maketitle

\section{Compound Interest}

Any financial agreement involving the borrowing of money typically includes an interest calculation. Interest can be thought of as the cost of borrowing from a lender and is usually computed as a percentage of the principal sum over a fixed period of time.

Compound interest involves the addition of the previously computed interest to the principal sum. That is, you compute the interest on the interest on the loan.

We use the following basic formula for compounding interest:

$$P' = P\cdot\left(1+\frac{r}{n}\right) + c $$ where
\begin{itemize}
	\item $P = $ principal sum,
	\item $r = $ annual interest rate,
	\item $n = $ number of compoundings per year,
	\item $c = $ contribution to the principal per compounding period, and 
	\item $P' = $ new principal sum after one compounding period.
\end{itemize}

When $c=0$, we can compute the new principal after $t$ compounding periods as $P\cdot\left(1+\frac{r}{n}\right)^{t}$. When $c\neq 0$, the following formulas for geometric sums will be helpful:

$$\sum_{k=0}^N r^k = \frac{1-r^{N+1}}{1-r}$$

and

$$\sum_{k=0}^N r^k = \frac{1}{1-r}, |r| < 1.$$

\section{Problems}

\begin{question}
Suppose you open a retirement account with \$1,000 that offers an annual interest rate of 6\% compounded monthly. (Round all answers to the nearest cent.)

How much is the account worth after 20 years with no monthly contributions? $\answer{3310.20}$

How much is the same account worth after 20 years if you make monthly contributions of \$50? $\answer{26412.25}$
\end{question}

\begin{question}

\end{question}

\section{Workspace}

\begin{sageCell}
# Use this cell to solve the above questions.
\end{sageCell}

\end{document}
