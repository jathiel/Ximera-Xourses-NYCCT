\documentclass{ximera}  
\title{Discrete Models}  
\begin{document}  
\begin{abstract}  
We provide a brief instroduction to discrete models.
\end{abstract}  
\maketitle

\section{Discrete Models}

The definition of a discrete set is beyond the scope of this class, so
instead we will illustrate the concept via examples and non-examples. Examples
of discrete (sub)sets of the real line include:
\begin{itemize}
	\item the integers and
	\item rationals with a denominator of 2 or 3.
\end{itemize}
Examples of (sub)sets of the real line that are not discrete include:
\begin{itemize}
	\item the real line,
	\item all rationals, and
	\item the open interval $(0,1)$.
\end{itemize}
One way of thinking of a discrete set is that its elements must be
``isolated" from each other. 

Discrete sets and models are important in 
this setting since we are using a computer, which cannot perfectly model
a continuous process. In many cases, we wish to model over discrete data
(like in a recurrence relation) or we may try to approximate a 
continuous model using a discrete one (this idea will be revisited in 
MAT4880 since, in some cases, such approximations do not work).

\section{Examples}

\subsection{A Population Model}

Suppose that observations of the local turtle population in a nearby 
lake show that the population follows the following trends:
\begin{itemize}
	\item When first observed, the population of turtles was 20
	in the year 2010.
	\item Every 5 years about half the turtles die off and 15
	new turtles are born.
\end{itemize}
What will be the long-term behavior of this population? What happens
to the population should some of our initial assumptions change?

\begin{sageCell}
# Turtle population problem
import matplotlib.pyplot as plt

x = [20]
r = 15 # try these values 0.75, 1.5, 2.3, 2.6, 3.1, 3.3, 3.5, 3.8
for i in range(100):
    x.append(0.5*x[-1]+r)
             
plt.scatter(range(101),x)
plt.show()
\end{sageCell}

\subsection{The Logistic Map}

The logistic map is a recurrence relation that is a discrete 
approximation to a commonly used population model. It is given by
\begin{align*}
	x_0 &= c\\
	x_{n+1} &= rx_n(1-x_n)
\end{align*}
for a pair of constants $c$ and $r$. Taking $c=0$, we can plot 
$x_0$ to $x_{100}$ for various choices of $r$ to see how it affects
the result. 

\begin{sageCell}
# An example of the logistic map
import matplotlib.pyplot as plt

x = [0.5]
r = 3.8 # try these values 0.75, 1.5, 2.3, 2.6, 3.1, 3.3, 3.5, 3.8
for i in range(100):
    x.append(r*x[-1]*(1-x[-1]))
             
plt.scatter(range(101),x)
plt.show()
\end{sageCell}

\subsection{Roulette}

The roulette wheel (see image below) is a gambling game where players
can bet on the colors black or red for an even payout (by this we mean
if you bet a dollar and win, you win a dollar). As can be seen from
the image, there are 38 spots on the roulette wheel, 18 black, 18 red,
and 2 green. In other words, the chances of winning a bet are $18/38$,
which is about $47.4\%$. So in the long run, a player should expect
to eventually lose all their money if they continue playing as long as
they still have money to spend.

Consider the following gambling `strategy'. Suppose a player starts
with \$20 and plans to make \$10 bets until they either double their
original starting amount or go broke. What are the odds of either 
scenario? Does it make sense to bet in smaller or larger increments?





\section{Compartment Models}




\end{document}
