\documentclass{ximera}  
\title{Monte Carlo Simulations}  
\begin{document}  
\begin{abstract}  
We give an introduction to Monte Carlo simulations using NumPy.
\end{abstract}  
\maketitle

\section{Monte Carlo Simulations}

Suppose you flip a fair coin 500 times. What is the length of the longest streak of consecutive heads that you expect to see? Computing this value exactly requires some knowledge of probability theory and a decent amount of work. 

As an alternative to computing the theoretically expected value, you might consider running your own experiment: flipping a coin 500 times and counting the length of the longest streak of consecutive heads. This experiment could take a while to complete just once and it should be reasonable to think that you are going to have to repeat this experiment many times to get a sense of what happens on average.

The alternative described above seems to solve the problem of having to have specialized knowledge, but doing the experiment by hand seems time consuming. Why not use the computer to simulate the result of this experiment thousands of times? This type of approach is sometimes called a Monte Carlo simulation.

Monte Carlo simulations can be useful when:
	\begin{itemize}
		\item computing the probability of an event is difficult or
		\item running a physical experiment is too costly or time consuming.
	\end{itemize}
Our general approach to using Monte Carlo simulations will involve three key steps:
	\begin{enumerate}
		\item Identify the situation or experiment to be modelled.
		\item Simulate the result of one experiment using a computer program.
		\item Repeat the previous step many times.
	\end{enumerate}
Usually step (b) is the most challenging part. Once we can simulate the experiment once, we can wrap all of the code inside a function and then loop it.

\section{Random Sampling}

In order to run a Monte Carlo simulation, we are going to have to generate random numbers somehow. The NumPy library can help us with this. (For a list of available functions related to random sampling, see this \link[link]{https://numpy.org/doc/stable/search.html?q=Random\%20sampling\%20(numpy.random)}.) We will cover some basic examples here.
\begin{enumerate}
	\item \verb|np.random.rand()| - This generates a random number using the uniform distribution in the interval $[0,1)$.
	\item \verb|np.random.ranint(a,b,size=N)| - This generates a random integer in the interval $[a,b)$. The optional argument \verb|size| specifies how many samples we want.
	\item \verb|np.random.choice(myList, p = prob_list)| - This generates a random selection from \verb|myList|. The optional argument \verb|p| allows you to specify the probability of selecting each item on the list.
\end{enumerate}

\section{Examples}



\begin{verbatim}
==============================

==============================
\end{verbatim}


\section{Problems}

\begin{question}
\end{question}

\begin{question}
\end{question}

\begin{question}
\end{question}

\section{Workspace}

\begin{sageCell}
# Use this cell to solve the above questions.
\end{sageCell}

\end{document}
