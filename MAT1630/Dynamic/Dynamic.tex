\documentclass{ximera}  
\title{Dynamic Programming}  
\begin{document}  
\begin{abstract}  
We give an introduction to dynamic programming techniques via examples.
\end{abstract}  
\maketitle

\section{Dynamic Programming}

In the Recursion section, we defined the Fibonacci sequence recursively as $$f_n=\begin{cases} 1 & \text{if $n=0,1$}\\ f_{n-1}+f_{n-2} & \text{otherwise.}\end{cases}$$ Additionally, we presented the following code to compute these values:
\begin{sageCell}
def fib(n):
        if n < 2:
                return 1
        else:
                return fib(n-1) + fib(n-2)			
fib(5)
\end{sageCell}
While the function for \verb|fib| is very short and easy to read, it does suffer from one major drawback. If you try to compute $f_{100}$ you will find that the computer will complain and not be able to do it. Why? Because to compute \verb|fib(100)|, the program will first attempt to compute \verb|fib(99)| before computing \verb|fib(98)| and adding those values together. But to compute \verb|fib(99)| it will first compute \verb|fib(98)| before computing \verb|fib(97)| and so on. So you can see that the program is giving the computer a lot of needless work. The program ends up computing the same values multiple times because it never ``remembers" that it already did the same work before. A better approach would be to start from the beginning and work our way up to the desired term in the sequence. As we work our way up, we write down, in a list or table, the terms of the sequence as we compute them. If we ever need them again, we just look to our table of computed values rather than recomputing them (which can waste a lot of time). Below is an example of this approach:
\begin{sageCell}
def dyn_fib(n):
        if n < 2:
                return 1
        else:
                L = [0 for i in range(n)]
                L[0] = 1
                L[1] = 1
                for i in range(2,n):
                        L[i] = L[i-1]+L[i-2]
                return L[-1]
\end{sageCell}



\section{Problems}

\begin{question}
\end{question}

\section{Workspace}

\begin{sageCell}
# Use this cell to solve the above questions.
\end{sageCell}

\end{document}
