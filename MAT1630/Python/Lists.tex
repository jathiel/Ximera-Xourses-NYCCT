\documentclass{ximera}  
\title{Lists and Strings}  
\begin{document}  
\begin{abstract}  
We give a brief discussion on lists, indexing, slicing, and strings.
\end{abstract}  
\maketitle

\section{Lists}

A list is an ordered sequence of comma separated values. Members of a list are often referred to as elements and are assigned a position, known as their index, on the list. Lists are zero-indexed, meaning that the first element in the list is indexed by 0, the second by 1, and so on.

Below we give an example of a list and how to reference various elements in the list.

\begin{verbatim}
==============================
a = [1, 3, 6, -1, 5]
==============================
\end{verbatim}

The variable \verb|a| above is a list containing five elements. The first element on the list can be referenced by \verb|a[0]|, the second by \verb|a[1]|, etc. In Python, one can use negative numbers as well to go backwards, making the last element \verb|a[-1]|, the penultimate \verb|a[-2]|, etc. Note that to create a list, the sequence of comma separated values must be enclosed by square brackets \verb|[ ]|. Parentheses and curly braces are for different, but related, data types that we will avoid for now.

Lists are useful in that they can hold different data types and one can iterate over them. It is this second property that will be explained further and used extensively in the next section. 

\section{Functions and Operators}

Python comes equipped with functions and operators designed to work in a particular way when applied to lists. Here are some common examples:

\begin{itemize}
	\item Concatenation (\verb|+|) - This operator concatenates lists, so \verb|[1,2]+[3,4]| would yield \verb|[1,2,3,4]|.
	\item Multiplication (\verb|*|) - This operator requires a list and integer and is best illustrated via an example. We have that \verb|[1,2] * 3| yields \verb|[1,2,1,2,1,2]|.
	\item Length (\verb|len|) - This function computes the length of a list, so \verb|len([1,3])| yields \verb|2|.
	\item Membership (\verb|in|) - This keyword can be used to determine if a particular value is a member of a list. For example, \verb|2 in [1,2,3]| is \verb|True|.
\end{itemize}

\section{Slicing}

List slicing is a handy way of obtaining a portion of a list using the following notation. Given a list \verb|a|, \verb|a[start:stop]| will give all of the elements of \verb|a| starting with index \verb|start| and ending with index \verb|stop-1|. If either \verb|start| or \verb|stop| are omitted, then the longest list satisfying the contraints is given. Also, a third optional argument can be given to specify the step size taken. We illustrate various ways to slice a list below.

\begin{verbatim}
==============================
myList = ['cake', 'pie', 'ice cream', 'donut', 'brownie', 'fruit']

print(myList[2:5])  # Contains elements with index 2,3,4.
print(myList[:3])   # Contains elements with index 0,1,2.
print(myList[1:])   # Contains elements with index >= 1.
print(myList[2::2]) # Contains every other element with index >= 2.
print(myList[::-1]) # Reverses the list.
==============================
\end{verbatim}

\section{Caution}

Consider the following example:

\begin{verbatim}
==============================
a = [1,2,3]

b = a

b[1] = -7

print(a) # What do you expect to see here?
print(b)
==============================
\end{verbatim}

Note that \verb|b = a| does not create a new list \verb|b| with the same elements as \verb|a|. Rather, both are references to the same object. Any changes to \verb|b| will therefore be reflected by \verb|a| as well. To create a duplicate copy of a list that avoids this issue, use \verb|b = a[:]|.

\section{Strings}

A character is a data type that is a single symbol such as a letter, number, or punctuation mark. Characters can be differentiated from other objects (like integers or variables) by their use of single or double quotes. For example, \verb|"a"| is a character, while \verb|a| is a variable.

A string is a sequence of characters. Strings can be used to read, process, and output text. Strings are similar to lists in that we can use much of what we have learned about lists for strings. See the examples below.

Many of the list operators we saw above also work for strings.

\begin{verbatim}
==============================
a = 'foo'
b = 'bar'

print(a+b)
print('|') # This is used to separate the results of each print statement.
print(a*3)
print('|')
print(len(a))
print('|')
print('f' in a)
==============================
\end{verbatim}

We give some examples of string slicing below.

\begin{verbatim}
==============================
myString = 'Hello, world!'

print(myString[2:5])  # Contains characters with index 2,3,4.
print('|')
print(myString[:3])   # Contains characters with index 0,1,2.
print('|')
print(myString[1:])   # Contains characters with index >= 1.
print('|')
print(myString[2::2]) # Contains every other character with index >= 2.
print('|')
print(myString[::-1]) # Reverses the string.
==============================
\end{verbatim}


\section{Problems}

\begin{question}
Consider the list \verb|a = [5,5,4,3,6]| and determine the values below.
	\begin{enumerate}
	\item \verb|a[0]| $\answer{5}$
	\item \verb|a[2]| $\answer{4}$
	\item \verb|a[3]| $\answer{3}$
	\item \verb|a[-1]| $\answer{6}$
	\item \verb|a[4]| $\answer{6}$
	\end{enumerate}
\end{question}

\begin{question}
	Consider the list \verb|a = [4,5,2,1,6]|. Use list slicing to obtain the following lists:
	\begin{enumerate}
	\item \verb|[2,1,6]|
	\item \verb|[4,5]|
	\item \verb|[4,2,6]|
	\item \verb|[6,1,2,5,4]|
	\end{enumerate}
	Note that the hint for this problem is the solution.
	\begin{hint}
\begin{verbatim}
==============================
a = [4,5,2,1,6]

print(a[2:])  # Solution to part a.
print(a[:2])  # Solution to part b.
print(a[::2]) # Solution to part c.
print(a[::1]) # Solution to part d.
==============================
\end{verbatim}
	\end{hint}
\end{question}

\begin{question}
	Define a function \verb|initials| that takes in two strings \verb|first| and \verb|last| as inputs, representing an individuals first and last name (respectively) and returns their initials. By initials we mean a string containing two characters, the first character of \verb|first| and the first character of \verb|last|. So, for example, if \verb|a="grace"| and \verb|b="hopper"|, then the output should be \verb|gh|.
	\begin{hint}
	Use the contatenation operator \verb|+| to concatenate the first character from \verb|first| and the first character from \verb|last|. Note that the second hint for this problem is the solution.
	\end{hint}
	\begin{hint}
\begin{verbatim}
==============================
def initials(a,b):
	return a[0] + b[0]
==============================
\end{verbatim}
	\end{hint}
\end{question}

\begin{question}
Define a function \verb|longer| that takes in two lists \verb|a| and \verb|b| as inputs and returns the length of the longer list. Note that the hint for this problem is the solution.
	\begin{hint}
\begin{verbatim}
==============================
def longer(a,b):
        if len(a) >= len(b):
                long = len(a)
        else:
                long = len(b)
        return long
==============================
\end{verbatim}
	\end{hint}
\end{question}

\begin{question}
A string is a palindrome if it is the same forwards and backwards, like \verb|racecar|. Define a function \verb|pal| that takes in a string \verb|a| as an input and returns \verb|True| if the string is a palindrome and \verb|False| otherwise.
	\begin{hint}
	If a string is a palindrome, how does it compare to its reversal? 
	\end{hint}
	\begin{hint}
	This problem can be done without an if/else statement. A comparison, such as \verb|x > 2| will be equal to \verb|True| or \verb|False| depending on the value of \verb|x|. That is, if we simply want to know if \verb|x| really is larger than 2, we can use \verb|return x > 2|. See the example below. Note that the third hint for this problem is the solution.
\begin{verbatim}
==============================
def greater_than_2(x):
        return x > 2

print(greater_than_2(3))
print(greater_than_2(0))
==============================
\end{verbatim}
	\end{hint}
	\begin{hint}
\begin{verbatim}
==============================
def pal(a):
        return a == a[::-1]
==============================
\end{verbatim}
	\end{hint}
\end{question}

\end{document}
