\documentclass{ximera}  
\title{Newton's Method}  
\begin{document}  
\begin{abstract}  
We give an introduction to finding roots using Newton's Method.
\end{abstract}  
\maketitle

\section{Linear Approximations}

In a calculus course, it is shown that the best linear approximation to a differentiable function $f(x)$ near $x=a$ is given by $L(x)=f'(a)(x-a)+f(a)$. We will make use of this linear approximation to try to approximate solutions of equations of the form $f(x)=0$. 

The Desmos interactive graph below gives an example of the IVT in action.

\desmos{uroxd0wmuh}{800}{600}


\section{Newton's Method}

Suppose that $f(x)$ is a differentiable function and that we would like to solve the equation $f(x)=0$. Using the IVT, or a graphical method, we might have an initial guess for a solution that we will call $x_0$. We are interested in finding a value $r$ such that $f(r)=0$ and we know that if $r$ and $x_0$ are ``close" to each other, then $$f(r)\approx f'(x_0)(x-x_0)+f(x_0).$$ Solving for $r$ in terms of $x_0$ and renaming it as $x_1$ gives that $$x_1=x_0-\frac{f(x_0)}{f'(x_0)}.$$ Iterating this process gives that $$x_n=x_{n-1}-\frac{f(x_{n-1})}{f'(x_{n-1})}$$ for $n\geq 1$.  


\begin{center}
	\includegraphics{bisection.png}
\end{center}

Converting this flowchart to code gives the following:

\begin{verbatim}
==============================
def bisection(a,b,f,error):
==============================
\end{verbatim}




\section{Problems}

All answers in this section have been rounded to the nearest thousandth.

\begin{question}
	Use the Bisection Method to estimate the solution to $f(x)=0$ where $f(x)=x^3-7$ with an error of at most 0.01. $\answer{1.914}$ 
\end{question}

\begin{question}
	Use the Bisection Method to give an approximation of $\sqrt{2}$ with an error of at most 0.01. $\answer{1.414}$
	\begin{hint}
		Find a simple function $f(x)$ such that $f(\sqrt{2})=0$ that is easy to work with.
	\end{hint}
	\begin{hint}
		Do not use $f(x)=x-\sqrt{2}$, as this would require an approximation of the number you wish to approximate.
	\end{hint}
\end{question}

\begin{question}
	Use the Bisection Method to give an approximation of the solution to the equation $x\ln{x}=4$ with an error of at most 0.01. $\answer{3.329}$
	\begin{hint}
		Remember to use \verb|import math| in order to use the logarithm.
	\end{hint}
\end{question}

\begin{question}
	Use the Bisection Method to give an approximation of all five of the solutions to $\ln{x}+\sin{x}=2$ on the interval $[5,20]$. Enter your solutions in increasing order. $\answer{6.424}$ $\answer{9.705}$ $\answer{12.061}$ $\answer{16.654}$ $\answer{17.779}$
	\begin{hint}
		Use a graphical tool (like Matplotlib) to get a sense for the locations of the solutions.
	\end{hint}
\end{question}

\section{Workspace}

\begin{sageCell}
# Use this cell to solve the above questions.
\end{sageCell}

\end{document}
