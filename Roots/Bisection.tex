\documentclass{ximera}  
\title{Bisection Method}  
\begin{document}  
\begin{abstract}  
We give an introduction to finding roots using the Bisection Method.
\end{abstract}  
\maketitle

\section{The Intermediate Value Theorem}

The Intermediate Value Theorem simply states that if $f(x)$ is a continuous function on the interval $[a,b]$, then it takes on any given value between $f(a)$ and $f(b)$. In other words, if $f(x)$ is a continuous function, then any horizontal line between $f(a)$ and $f(b)$ intersects the graph of $f(x)$. Note that this theorem does not give the number or location of such intersections.

\begin{tikzpicture}
\begin{scope}
\clip (-3,-2) rectangle (3,2);
\draw[thick,smooth,domain=-3:3] plot (\x,{\x^3/3 - \x});
\end{scope}
\node[point,fill=black] (a) at (-2,-2/3) {};
\node[point,fill=black] (b) at (2,2/3) {};
\coordinate (origin) at (-4,-3);
\coordinate (topright) at (4,2);
\draw[<->] (topright -| origin) -- (origin) -- (origin -| topright);
\draw[dotted,very thick] (a) -- (a|-origin) node[below] {$a$};
\draw[dotted,very thick] (b) -- (b|-origin) node[below] {$b$};
\end{tikzpicture}

\section{Problems}

Note that for the questions below, the hint contains the solution.

\begin{question}
\end{question}

\section{Workspace}

\begin{sageCell}
# Use this cell to solve the above questions.
\end{sageCell}

\end{document}
